\documentclass{article}
\usepackage[utf8]{inputenc}
\usepackage{mathtools}
\usepackage{soul}
\usepackage{float}
\usepackage{listings}
\usepackage{tabularx}

% Arrows.
\usepackage{mathabx}

% Programming languages
% https://mirror.its.dal.ca/ctan/macros/latex/contrib/semantic/semantic.pdf
\usepackage{semantic}

\title{Types and Programming Languages (Pierce)}
\author{Bruno Flores}
\date{September 2021}

\begin{document}

\maketitle

\section{The Untyped Lambda-Calculus}

\subsection{Syntax}

The syntax of the lambda-calculus comprises just three sorts of terms:

\begin{table}[h!]
\begin{tabularx}{1\textwidth} 
  { >{\raggedright\arraybackslash}l
    >{\raggedright\arraybackslash}l
    >{\raggedleft\arraybackslash}X }

    \texttt{x ::=} & & terms: \\
    & \texttt{x} & variable \\
    & \(\lambda\)\texttt{x.t} & abstraction \\
    & \texttt{t t} & application \\
\end{tabularx}
\caption{Syntax}
\end{table}

\subsection{The \textit{call-by-value} strategy}

\begin{table}[h!]
\begin{tabularx}{1\textwidth} 
  { >{\raggedright\arraybackslash}l
    >{\raggedright\arraybackslash}l }

    & id \underline{(id (\(\lambda\)z. id z))} \\
    \(\rightarrow\) & \underline{id (\(\lambda\)z. id z)} \\
    \(\rightarrow\) & \(\lambda\)z. id z \\
    \(\nrightarrow\) & \\
\end{tabularx}
\end{table}

The call-by-value strategy is \textit{strict}, in the sense that the arguments to functions are always evaluated, whether or not they are used by the body of the function.

\subsection{Variable substitution}

\paragraph{Intuition} The names of bound variables do not matter.

\paragraph{Convention} Terms that differ only in the names of bound variables are interchangeable in all contexts.

In practice, the name of any \(\lambda\)-bound variable can be changed to another name (consistently making the same change in the body of the \(\lambda\)), at any point where this is convenient.

\begin{table}[h!]
\begin{tabularx}{1\textwidth} 
  {   >{}X
    | >{}X }

    Syntax & Evaluation 
    \\\\
    {
    \begin{tabularx}{1\linewidth}
    { >{\raggedright\arraybackslash}l 
      >{\raggedright\arraybackslash}l
      >{\raggedleft\arraybackslash}X }
      
      \texttt{t ::=} & & terms: \\
      & \texttt{x} & variable \\
      & \texttt{\(\lambda\)x.t} & abstraction \\
      & \texttt{t t} & application \\
      \\
      \texttt{v ::=} & & values: \\
      & \texttt{\(\lambda\)x.t} & abstraction value \\
      
    \end{tabularx}
    }
    &
    {
    \begin{tabularx}{1\linewidth}
    { >{\raggedright\arraybackslash\texttt}l 
      >{\raggedleft\arraybackslash}X }
      
      \inference[]
      {\texttt{t$_1$ $\rightarrow$ t$'_1$}}
      {\texttt{t$_1$ t$_2$ $\rightarrow$ t$'_1$ t$_2$}} & (E-App1)
      \\\\
      \inference[]
      {\texttt{t$_2$ $\rightarrow$ t$'_2$}}
      {\texttt{v$_1$ t$_2$ $\rightarrow$ v$_1$ t$'_2$}} & (E-App2)
      \\\\
      \texttt{($\lambda$x.t$_{12}$)v$_2$ $\rightarrow$ 
      [x $\mapsto$ v$_2$]t$_{12}$} & (E-AppAbs)
      
    \end{tabularx}
    }

\end{tabularx}
\caption{Untyped lambda-calculus}
\label{untyped}
\end{table}

The evaluation relations in the right-hand column of Table \ref{untyped} completely determine the order of evaluation for an application \texttt{t$_1$ t$_2$}: we first use E-App1 to reduce \texttt{t$_1$} to a value, then use E-App2 to reduce \texttt{t$_2$} to a value, and finally use E-AppAbs to perform the application itself.

\subsubsection{de Bruijn presentation}

\end{document}